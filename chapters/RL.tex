\section{Reinforcement Learning}
\emph{Unknown $r(x,a)$, $P(x'|x,a)$.}
\subsection{Model-based, $|A|$ small, $|X|$ small}
\mbox{1. Est $P(x'|x,a) = \frac{\#(x',x,a)}{\#(x,a)}$,
$r(x,a) = \frac{1}{\#(x,a)}\smashoperator{\sum_{t:(x_t,a_t)=(x,a)}}r_t$.}\\
2. Update $\pi$.\\
3. Explore \& exploit:\\
3.1 ($\color{black}\mathbf{\epsilon}$\textbf{-greedy}): $a_t=\mathrm{Cat}_{[1-\epsilon_t, \epsilon_t]}[\pi^*(x_t),\mathrm{rand}]$. CV a.s. if $\sum \epsilon_t = \infty$, $\sum \epsilon^2_t < \infty$.\\
3.2 ($\color{black}\mathbf{R_{\max}}$): Set $r(x,a)=R_{\max}$, $P(x^*|x,a)=1$, $r(x^*,a)=R_{\max}$. Update with empirical estimates after collecting $n_{x,a} \geq \frac{R_{\max}^2}{2\epsilon^2}\log\frac{2}{\delta}$ samples. \emph{Every $T$ steps, w.h.p. $R_{\max}$ yields near-opt. reward or visits $\geq 1$ unknown state-action pair.\\}
\emph{With $p=1-\delta$ $R_{\max}$ reaches $\epsilon$-opt. policy in $O(\mathrm{poly}(|X|,|A|,T,\frac{1}{\epsilon},\log\frac{1}{\delta}, R_{\max})$)}
\subsection{Model-free, $|A|$ small, $|X|$ small}
\textbf{TD-Learning:} \emph{Goal:} Find $V^\pi$ given $\pi$.\\
\emph{Algorithm:} 1. Follow $\pi$ to get $(x,a,r,x')_t$. 2.Update $\hat{V}^\pi(x) \leftarrow (1-\alpha_t)\hat{V}^\pi(x_t) + \alpha_t (r + \gamma\hat{V}^\pi(x_t'))$.
\emph{If $\sum\alpha_t=\infty$, $\sum\alpha_t^2<\infty$ and every (x,a) is visited $\color{black}\infty$-ly often, then a.s. $\hat{V}^\pi \to V^\pi$.}\\
\textbf{{\color{gray} Optimistic} Q-Learning:} {\color{gray} \emph{\textbf{Init:}} $\color{gray}\hat{Q}^*(x,a) = \frac{R_{\max}}{1-\gamma}\prod_{t=1}^{T_\mathrm{init}}(1-\alpha_t)^{-1}$}.
\emph{\color{gray}\textbf{Explore:}} $\color{gray}a_t \in \argmax_a \hat{Q}^*(x,a)$. \emph{\textbf{Update:}} $\hat{Q}^*(x_t,a_t) \leftarrow (1-\alpha_t)\hat{Q}^*(x_t,a_t) + \alpha_t(r_t + \gamma(\max_a\hat{Q}^*(x'_t,a)$.\\
\emph{If $\sum\alpha_t=\infty$, $\sum\alpha_t^2<\infty$ and every (x,a) is visited $\color{black}\infty$-ly often, then a.s. $\hat{Q}^* \to Q^*$.}\\
\emph{\color{gray} With $\color{gray}p=1-\delta$ Optimistic Q-Learning reaches $\epsilon$-opt. policy in $O(\mathrm{poly}(|X|,|A|,T,\frac{1}{\epsilon},\log\frac{1}{\delta})$)}
\subsection{Model-free, $|A|$ small, $|X|$ large}
\textbf{Deep Q-Learning:} Define $Q(x,a;\theta) \approx Q^*(x,a)$.
\textbf{\emph{Training:}} Minimize $ l_2(x,a,r,x',\theta) = \frac{1}{2}\delta^2$ with $\delta = Q(x,a;\theta) - (r+\gamma\max_{a'}Q(x',a';\theta_\mathrm{old}))$.\\
\emph{Update:} $\theta_{t+1} \leftarrow \theta_t - \alpha_t\delta\nabla_{\theta_t}Q(x,a;\theta_t)$.\\
\textbf{Experience Replay:} Collect $D=\{(x,a,r,x')\}$, fix $\theta_\mathrm{old}$. Minimize $L(\theta) = \sum_{d\in D}l_2(d,\theta)$.
\textbf{Double DQN:} Reduce \textbf{maximization bias} by changing $\max_{a'}Q(x',a';\theta_\mathrm{old}))$ in $\delta$ to $\max_{a'}Q(x',a';\theta))$. 
\subsection{Model-based RL}

\subsubsection{$\epsilon$ greedy}
With probability $\epsilon$, pick random action. With prob $(1-\epsilon)$, pick best action. If sequence $\epsilon$ satisfies Robbins Monro criteria $\rightarrow$ convergence to optimal policy with prob 1.

\subsubsection{$R_{max}$ algorithm}
\textbf{Input}: starting $x_0$, discount factor $\gamma$.\\
\textbf{Initially}: add fairy tale state $x^*$ to MDP\\
- Set $r(x,a)=R_{max}$ for all states x and actions $a$\\
- Set $P(x^*|x,a)=1$ for all states $x$ and actions $a$\\
- Choose the optimal policy for $r$ and $P$\\
\textbf{Repeat}:
1. Execute policy $\pi$ and, for each visited state/action pair, update $r(x,a)$\\
2. Estimate transition probabilities $P(x^{'}|x,a)$\\
3. If observed 'enough' transitions/rewards, recompute policy $\pi$, according to current model $P$ and $r$.\\
\textbf{"Enough"?} See Hoeffding's inequality. To reduce error $\epsilon$, need more samples $N$.\\
\textbf{Theorem}: With probability $1-\delta$, $R_{max}$ will reach an $\epsilon$-optimal policy in a number of steps that is polynomial in $|X|, |A|, T, 1/\epsilon$ and $log(1/\delta)$. Memory $O(|X^2||A|)$. 

\subsection{Model-free RL: estimate V*(x) directly}
\subsubsection{Q-learning}
$Q(x,a) \leftarrow (1-\alpha_t)Q(x,a) + \alpha_t(r+\gamma \max_{a'}Q(x', a'))$\\
% $V^*(x)=\underset{a}{max}Q*(x,a)$\\
\textbf{Theorem}: If learning rate $\alpha_t$ satisfies: $\sum_t \alpha_t=\infty$ and $\sum_t \alpha_t^2 < \infty$ (Robbins-Monro), and actions are chosen at random, then $Q$ learning converges to optimal $Q^*$ with probability 1.\\
\textbf{Optimistic Q learning:}\\
Initialize: $Q(x,a)=\frac{R_{max}}{1-\gamma}\prod_{t=1}^{T_{init}}(1-\alpha_t)^{-1}$\\
Same convergence time as with $R_{max}$. Memory $O(|X||A|)$. Comp: $O(|A|)$.\\
\textbf{Parametric Q-function approximation}: $Q(x,a;\theta)=\theta^T\phi (x,a)$ to scale to large state spaces. (You can use Deep NN here!)\\
\textbf{SGD for ANNs}: initialize weights. For t = 1,2..., pick a data point (x,y) uniformly at random. Take step in negative gradient direction. (In practise, mini-batches).\\
\textbf{Deep Q Networks}: use CNN to approx Q function.
% Use "experience relay". Clone network to maintain constant "target" values across episodes:
$ L(\theta)=\sum_{(x,a,r,x')\in D}(r+\gamma\underset{a'}{max}Q(x',a';\theta^{old})-Q(x,a;\theta))^2$ \textbf{Double DQN:} current network for evaluating argmax (too optimistic, and you remove $\theta^{old}$ and put $\theta$).

\subsection{Gaussian processes}
A GP is an (infinite) set of random variables (RV), indexed by some set X, i.e., for each x in X, there is a RV $Y_x$ where there exists functions $\mu : X \rightarrow \mathbb{R}$ and $K: X \times X \rightarrow \mathbb{R}$ such that for all: $A \in X, \quad A={x_1,...x_k}$, it holds that $Y_A=[Y_{x_1},...,Y_{x_k}] \sim N(\mu_a, \Sigma_{AA})$, where: $\Sigma_{AA} =$ matrix with all combinations of $K(x_i, x_j)$.
% \[ 
% \left (
%   \begin{tabular}{cccc}
%   K(x_1,x_1) & K(x_1,x_2) & \cdots & K(x_1,x_n) \\
%   \vdots &  &  & \vdots \\
%   K(x_k,x_1) & K(x_k,x_2) & \cdots &K(x_k,x_k)
%   \end{tabular}
% \right )
% \]

% $\mu_A =$\begin{pmatrix}\mu(x_1)\\\mu(x_2)\\\vdots\\\mu(x_k}\end{pmatrix}
K is called kernel (covariance) function (must be symmetric and pd) and $\mu$ is called mean function.
\textbf{Making prediction with GPs:} Suppose $P(f)=GP(f;\mu, K)$ and we observe $y_i=f(\overrightarrow{x_i})+\epsilon_i$, $A=\{\overrightarrow{x_1}:\overrightarrow{x_k}\}$
$P(f(x)|\overrightarrow{x_1}:\overrightarrow{x_k},y_{1:k})=GP(f;\mu ', K')$.  In particular, $P(f(x)|\overrightarrow{x_1}:\overrightarrow{x_k},y_{1:k})=N()f(x);\mu_{x|A}, \sigma^2_{x|a}$, where $\mu_{x|a}=\mu(\overrightarrow{x})+\Sigma_{x,A}(\Sigma_{AA}+\sigma^2I)^{-1}\Sigma^T_{x,A}(\overrightarrow{y_A}-\mu_A)$ and $\sigma^2_{x|a}=K(\overrightarrow{x},\overrightarrow{x})-\Sigma_{x,A}(\Sigma_{AA}+\sigma^2I)^{-1}\Sigma^T_{x,A}$. \textbf{Closed form formulas for prediction!}





